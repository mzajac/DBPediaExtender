\documentclass[12pt]{article}

\usepackage{polski}

\author{Marcin Zając\\Institute of Computer Science, Polish Academy of Sciences}
\title{DBPedia-extender manual}

\begin{document}
    \maketitle
    \section{About}
    DBPedia-extender is a information extraction system 
    
    \section{Licensing}
    DBPedia-extender is released under the GNU General Public License v3.

    \section{Prerequisites}
    Note: the system was tested under versions listed. It may (and probably will) also work under different versions.\\
    Software:
    \begin{itemize}
        \item Python 2.6 or 2.7
        \item NTLK 2 (with support for Mallet).
        \item scikit-learn 0.11
        \item OpenLink Virtuoso (Open-Source Edition) 6.1.4
    \end{itemize}
    Data:
    \begin{itemize}
        \item DBPedia dumps\\
            Polish DBPedia does not give access to its dumps.
            However its creater, Knowledge Hives, when asked.
        \item Wikipedia dumps\\
            Available on http://dumps.wikimedia.org/plwiki/latest/, only pages-articles archive is necessary.\\
            Note that the Wikipedia dumps should be the same dumps from which DBPedia was extracted.
    \end{itemize}
    
    \section{Installation}
    DBPedia-extender is a pure python system, therefore it doesn't require any installation.
    However it relies on availability of two resources: DBPedia and Wikipedia articles.
    \subsection{DBPedia}
    
    \subsection{Wikipedia}
    
    
    \section{Usage}
    
    
\end{document}


